\chapter[PENDAHULUAN]{\\ PENDAHULUAN}

\section{Latar Belakang}
Latar belakang laporan proyek akhir sarjana terapan adalah latar yang menjelaskan tentang alasan dan dasar pemilihan topik proyek akhir, serta permasalahan atau masalah yang akan diteliti dalam proyek tersebut. Latar belakang ini juga menjelaskan tentang bagaimana proyek tersebut dapat memberikan solusi atau kontribusi terhadap permasalahan yang ada.

Proyek akhir sarjana terapan merupakan salah satu syarat untuk menyelesaikan pendidikan sarjana terapan. Proyek ini ditujukan untuk mengaplikasikan ilmu yang didapat dari perkuliahan ke dalam suatu proyek yang sesuai dengan bidang keahlian seseorang. Proyek akhir sarjana terapan juga dapat memberikan kontribusi bagi perkembangan ilmu pengetahuan dan teknologi dalam bidang yang diteliti.

Pemilihan topik proyek akhir sarjana terapan harus sesuai dengan minat dan bidang keahlian seseorang, serta harus memenuhi syarat yang ditentukan oleh institusi pendidikan. Topik yang dipilih harus memiliki permasalahan yang nyata dan dapat memberikan solusi atau kontribusi yang signifikan bagi bidang yang diteliti.

Dalam laporan proyek akhir sarjana terapan, diharapkan dapat diperoleh hasil yang valid dan dapat diuji kembali melalui metode yang sesuai. Hasil yang diperoleh dari proyek ini juga harus dapat memberikan solusi atau kontribusi yang bermanfaat bagi perkembangan ilmu pengetahuan dan teknologi dalam bidang yang diteliti.

Secara keseluruhan, latar belakang laporan proyek akhir sarjana terapan adalah untuk menjelaskan alasan dan dasar pemilihan topik proyek akhir, serta permasalahan atau masalah yang akan diteliti dalam proyek tersebut, serta memberikan solusi atau kontribusi yang bermanfaat bagi perkembangan ilmu pengetahuan dan teknologi dalam bidang yang diteliti.


\section{Rumusan Masalah}
Rumusan masalah dalam laporan proyek akhir sarjana terapan adalah bagian dari laporan yang menjelaskan secara spesifik dan jelas tentang permasalahan atau masalah yang akan diteliti dalam proyek tersebut. Rumusan masalah ini harus dapat diuraikan dengan baik dan jelas sehingga dapat diketahui apa yang akan diteliti dalam proyek akhir sarjana terapan tersebut.

Rumusan masalah dalam laporan proyek akhir sarjana terapan harus ditulis dengan menggunakan kalimat yang jelas dan spesifik. Rumusan masalah harus menjawab pertanyaan "apa yang akan diteliti dalam proyek ini?". Rumusan masalah juga harus memuat permasalahan yang akan diteliti, serta tujuan yang ingin dicapai dari proyek tersebut.

Contoh rumusan masalah dalam laporan proyek akhir sarjana terapan :
"Permasalahan yang akan diteliti dalam proyek ini adalah pengaruh perubahan iklim terhadap produktivitas tanaman padi di wilayah X. Tujuan dari proyek ini adalah untuk mengetahui pengaruh perubahan iklim terhadap produktivitas tanaman padi dan untuk mengetahui cara-cara untuk meningkatkan produktivitas tanaman padi di wilayah X yang terkena dampak perubahan iklim."

Secara keseluruhan, rumusan masalah dalam laporan proyek akhir sarjana terapan adalah bagian dari laporan yang menjelaskan secara jelas dan spesifik tentang permasalahan atau masalah yang akan diteliti dalam proyek tersebut, yang merupakan dasar dari penelitian yang akan dilakukan.

\section{Penelitian Terkait}
Bagian penelitian terkait dalam bab 1 pendahuluan pada laporan proyek akhir sarjana terapan adalah bagian yang menjelaskan tentang studi yang telah dilakukan oleh peneliti sebelumnya yang berhubungan dengan topik yang diteliti dalam proyek akhir sarjana terapan. Bagian ini juga menjelaskan tentang keterkaitan antara hasil penelitian yang telah dilakukan dengan proyek akhir sarjana terapan yang akan dilakukan.

Dalam bagian penelitian terkait, harus dijelaskan tentang studi yang telah dilakukan sebelumnya yang berhubungan dengan topik yang diteliti dalam proyek akhir sarjana terapan, seperti :

\begin{packed_enum}
    \item Judul penelitian
    \item Nama peneliti
    \item Tahun penelitian
    \item Metode penelitian
    \item Hasil penelitian
\end{packed_enum}

Bagian ini juga harus menjelaskan tentang keterkaitan antara hasil penelitian yang telah dilakukan dengan proyek akhir sarjana terapan yang akan dilakukan. Ini akan membantu dalam menjelaskan alasan mengapa proyek akhir sarjana terapan ini penting dan bagaimana proyek ini akan menambah atau menyempurnakan penelitian yang telah dilakukan sebelumnya.

Contoh bagian penelitian terkait dalam bab 1 pendahuluan pada laporan proyek akhir sarjana terapan:
"Beberapa studi telah dilakukan sebelumnya tentang pengaruh perubahan iklim terhadap produktivitas tanaman padi. Penelitian yang dilakukan oleh (nama peneliti) pada tahun (tahun penelitian) menunjukkan bahwa perubahan iklim menyebabkan penurunan produktivitas tanaman padi di wilayah (wilayah penelitian). Penelitian yang dilakukan oleh (nama peneliti) pada tahun (tahun penelitian) menunjukkan bahwa penerapan teknik (teknik yang diterapkan) dapat meningkatkan produktivitas tanaman padi di wilayah yang terkena dampak perubahan iklim. Proyek akhir sarjana terapan ini akan mengevaluasi pengaruh perubahan iklim terhadap produktivitas tanaman padi di wilayah X dan mencari cara-cara untuk meningkatkan produktivitas tanaman padi di wilayah tersebut."

Secara keseluruhan, bagian penelitian terkait dalam bab 1 pendahuluan pada laporan proyek akhir sarjana terapan adalah bagian yang menjelaskan tentang studi yang telah dilakukan oleh peneliti sebelumnya yang berhubungan dengan topik yang diteliti dalam proyek akhir sarjana terapan. Bagian ini bertujuan untuk memberikan gambaran tentang kondisi saat ini dari topik yang diteliti dan membantu dalam menjelaskan alasan mengapa proyek akhir sarjana terapan ini penting dan bagaimana proyek ini akan menambah atau menyempurnakan penelitian yang telah dilakukan sebelumnya. Dengan mengetahui hasil penelitian yang telah dilakukan sebelumnya, peneliti dapat membuat rencana yang lebih baik dan fokus dalam meneliti masalah yang diangkat dalam proyek akhir sarjana terapan.

\section{Tujuan}
Tujuan dalam laporan proyek akhir sarjana terapan adalah bagian yang menjelaskan tentang sasaran yang ingin dicapai dari proyek akhir sarjana terapan yang akan dilakukan. Tujuan ini harus jelas, spesifik, dan dapat diukur. Tujuan dalam laporan proyek akhir sarjana terapan harus menjawab pertanyaan "apa yang ingin dicapai dari proyek ini?"

Tujuan dalam laporan proyek akhir sarjana terapan harus ditulis dengan menggunakan kalimat yang jelas dan spesifik. Tujuan harus dapat diukur dan dapat dicapai melalui metode yang digunakan dalam proyek akhir sarjana terapan. Tujuan juga harus memuat permasalahan yang akan diteliti dan solusi yang akan diberikan melalui proyek akhir sarjana terapan tersebut.

Contoh tujuan dalam laporan proyek akhir sarjana terapan:
"Tujuan dari proyek ini adalah untuk mengetahui pengaruh perubahan iklim terhadap produktivitas tanaman padi di wilayah X dan untuk mengetahui cara-cara untuk meningkatkan produktivitas tanaman padi di wilayah X yang terkena dampak perubahan iklim."

Secara keseluruhan, tujuan dalam laporan proyek akhir sarjana terapan adalah bagian yang menjelaskan tentang sasaran yang ingin dicapai dari proyek akhir sarjana terapan yang akan dilakukan. Tujuan harus jelas, spesifik, dan dapat diukur, serta dapat diperoleh melalui metode yang digunakan dalam proyek akhir sarjana terapan. Tujuan juga harus memuat permasalahan yang akan diteliti dan solusi yang akan diberikan melalui proyek akhir sarjana terapan tersebut.


\section{Batasan Masalah}
Batasan masalah dalam laporan proyek akhir sarjana terapan adalah bagian yang menjelaskan tentang batasan atau keterbatasan dari permasalahan yang diteliti dalam proyek akhir sarjana terapan. Batasan masalah ini harus jelas dan spesifik agar dapat membatasi permasalahan yang diteliti dalam proyek tersebut.

Batasan masalah dalam laporan proyek akhir sarjana terapan harus menjelaskan tentang wilayah atau area yang diteliti, jenis data atau sumber data yang digunakan, metode yang digunakan, serta waktu yang digunakan dalam proyek akhir sarjana terapan.

Contoh batasan masalah dalam laporan proyek akhir sarjana terapan:
"Batasan masalah dalam proyek ini adalah pengaruh perubahan iklim terhadap produktivitas tanaman padi di wilayah X saja. Data yang digunakan dalam proyek ini hanya data yang diperoleh dari observasi lapangan dan wawancara dengan petani tanaman padi di wilayah X. Metode yang digunakan dalam proyek ini hanyalah observasi lapangan dan analisis statistik. Waktu yang digunakan dalam proyek ini adalah selama satu musim tanam."

Secara keseluruhan, batasan masalah dalam laporan proyek akhir sarjana terapan adalah bagian yang menjelaskan tentang batasan atau keterbatasan dari permasalahan yang diteliti dalam proyek akhir sarjana terapan. Batasan masalah harus jelas dan spesifik agar dapat membatasi permasalahan yang diteliti dalam proyek tersebut, seperti wilayah atau area yang diteliti, jenis data atau sumber data yang digunakan, metode yang digunakan, serta waktu yang digunakan dalam proyek akhir sarjana terapan. Ini akan membantu dalam menjelaskan batasan dari proyek yang akan dilakukan dan membuat proyek lebih fokus dalam penelitian.


\section{Sistematika Penulisan}
Sistematika penulisan adalah susunan atau struktur dari laporan proyek akhir sarjana terapan yang menjabarkan bagian-bagian yang harus ada dalam laporan proyek akhir sarjana terapan. Sistematika penulisan dapat berbeda antara satu institusi dengan institusi lainnya, namun umumnya terdiri dari beberapa bagian yang wajib ada, seperti :

\begin{packed_item}
    \item Bab 1 Pendahuluan
    \item Bab 2 Tinjauan Pustaka
    \item Bab 3 Desain dan Implementasi
    \item Bab 4 Hasil dan Pembahasan
    \item Bab 5 Kesimpulan dan Saran
    \item Daftar Pustaka
\end{packed_item}

Penjelasan detail dari masing-masing bab adalah sebagai berikut:

\begin{packed_enum}
    \item Bab 1 Pendahuluan : menjelaskan latar belakang, rumusan masalah, tujuan, batasan masalah, serta sistematika penulisan dari laporan proyek akhir sarjana terapan.
    \item Bab 2 Tinjauan Pustaka : menjabarkan tentang studi yang telah dilakukan oleh peneliti sebelumnya yang berhubungan dengan topik yang diteliti dalam proyek akhir sarjana terapan, serta membahas teori yang relevan dengan masalah yang akan diteliti. Bab ini berisi tentang kajian pustaka yang diperoleh dari berbagai sumber yang terkait dengan masalah yang akan diteliti.
    \item Bab 3 Desain dan Implementasi: menjabarkan tentang rencana dan perencanaan yang digunakan dalam melakukan penelitian dan pelaksanaan penelitian sesuai dengan rencana yang telah ditetapkan dalam desain penelitian. Desain penelitian terdiri dari beberapa elemen, seperti desain penelitian, metode pengumpulan data, sampel, dan analisis data. Implementasi meliputi tahap-tahap dari pelaksanaan penelitian, seperti pengambilan sampel, pengumpulan data, dan analisis data.
    \item Bab 4 Hasil dan Pembahasan: menjabarkan hasil yang diperoleh dari proyek akhir sarjana terapan dan memberikan pembahasan yang mendalam terkait dengan hasil tersebut. Bab ini juga berisi tentang interpretasi data yang diperoleh dari penelitian.
    \item Bab 5 Kesimpulan dan Saran: menjabarkan kesimpulan yang diperoleh dari proyek akhir sarjana terapan serta saran yang diberikan untuk penelitian selanjutnya.
    \item Daftar Pustaka : menjabarkan sumber-sumber yang digunakan dalam laporan proyek akhir sarjana terapan.
\end{packed_enum}

Secara keseluruhan, sistematika penulisan dalam laporan proyek akhir sarjana terapan adalah susunan atau struktur dari laporan proyek akhir sarjana terapan yang menjabarkan bagian-bagian yang harus ada dalam laporan proyek akhir sarjana terapan, yang meliputi Pendahuluan, Tinjauan Pustaka, Metode Penelitian, Hasil dan Pembahasan, Kesimpulan dan Saran, serta Daftar Pustaka. Sistematika penulisan yang baik akan membuat laporan proyek akhir sarjana terapan lebih mudah untuk dibaca dan dipahami.